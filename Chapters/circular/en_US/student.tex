\chapter{Circular Motion}

Let's say you tie a 0.16 kg billard ball to a long string and begin to swing
it around in a circle above your head. Let's say the string is 3
meters long, and the ball returns to where it started every 4
seconds. If you start your stopwatch as the ball crosses the
$x$-axis, the position of the ball at any time $t$ given by:

$$p(t) = [3 \cos{\left( \frac{2 \pi} {4}t\right)}, 3 \sin{ \left( \frac{2 \pi}{4}t\right) }, 2]$$

(This assumes that the ball would be going counter-clockwise if viewed
from above. The spot you are standing on is considered the origin $[0, 0, 0]$.)

Notice that the height is a constant -- 2 meters in this
case. That isn't very interesting, so we will talk just about the
first two components.  Here is what it would look like from above:

% 3sin = 1.267854785222098
% 3cos = 2.71892336110995
\begin{tikzpicture}[declare function={angle=25;},bullet/.style={inner
    sep=1pt,fill,draw,circle,solid}, scale=1.7]
    % Axis
    \draw[thick,-stealth,black] (-3.2,0)--(3.2,0) node[right] {$x$}; % x axis
    \draw[thick,-stealth,black] (0,-3.2)--(0,3.2) node[left] {$y$}; % y axis
    % Rest
    \draw [dashed, sdkblue] (0,0) circle (3);
    \draw[thick] (0,0) -- (angle:3.0) node [midway, above] {3};
    \draw[sdkblue] (1.05, 0.2) node[right] {$\theta = \frac{2\pi t}{4}\text{ radians}$};
    \draw[-stealth,sdkblue] (1,0) arc (0:angle:1);
    \draw[dashed, black] (2.71892336110995, 1.267854785222098) -- (2.71892336110995, 0)
    node[below] {$3 \cos(\theta)$}; % vertical
    \draw[dashed, black] (2.71892336110995, 1.267854785222098) -- (0, 1.267854785222098)
    node[left] {$3 \sin(\theta)$}; % horizontal
    \filldraw[black] (angle:3.0) circle(4pt);
    \draw[->, thick] (2.71892336110995, 1.267854785222098) --
    (2.71892336110995 - 3 * 0.1267, 1.267854785222098 + 3 * 0.2718);
\end{tikzpicture}

In this case, the radius, $r$, is 3 meters.  The period, $T$ is 4
seconds.  In general, we say that circular motion is given by:

$$p(t) = \left[ r \cos{\frac{2 \pi t}{T}}, r \cos{\frac{2 \pi t}{T}}\right]$$

A common question is ``How fast is it turning right now?''  If you
divide the $2\pi$ radians of a circle by the 4 seconds it takes, you
get the answer ``About 1.57 radians per second.''  This is known as
\newterm{angular velocity} and we typically represent it with the
lowercase Omega: $\omega$. (Yes, it looks a lot like a ``w''.)  To be
precise, in our example, the angular velocity is $\omega = \frac{\pi}{2}$.

Notice that this is different from the question ``How fast is it
going?''  This ball is traveling the circumference of $6\pi \approx
18.85$ meters every 4 seconds.  So the speed of the ball is about
4.71 meters per second.

\section{Velocity}

The velocity of the ball is a vector, and we can find that vector by
differentiating each component of the position vector.

For any constants $a$ and $b$:

\begin{tabular}{c | c }
  Expression & Derivative \\
  \hline
  $a \sin{b t}$ & $ab \cos{b t}$ \\
  $a \cos{b t}$ & $-ab \sin{b t}$  \\
\end{tabular}

Thus, in our example, the velocity of the ball at any time $t$ is given by:

$$v(t) = \left[ -\frac{3 (2\pi)}{4} \sin{\frac{2\pi t}{4}}, \frac{3(2\pi)}{4} \cos{\frac{2\pi t}{4}}, 0 \right]$$

Notice that the velocity vector is perpendicular to the position vector.  It has a constant magnitude.

In general, an object traveling in a circle at a constant speed has the velocity vector:

$$v(t) = \left[ -r\omega \sin{\omega t}, r\omega \cos{\omega t}\right]$$

where $t = 0$ is the time that it crosses the $x$ axis.  If \omega is
negative, that means the motion would be clockwise when viewed from
above.

The magnitude of the velocity vector is $r\omega$. 

\begin{tikzpicture}[declare function={angle=25;},bullet/.style={inner
    sep=1pt,fill,draw,circle,solid}, scale=1.7]
    % Axis
    \draw[thick,-stealth,black] (-3.2,0)--(3.2,0) node[right] {$x$}; % x axis
    \draw[thick,-stealth,black] (0,-3.2)--(0,3.2) node[left] {$y$}; % y axis
    % Rest
    \draw [dashed, sdkblue] (0,0) circle (3);
    \filldraw[black] (angle:3.0) circle(4pt);
    \draw[->, thick] (2.71892336110995, 1.267854785222098) --
    (2.71892336110995 - 0.6 * 1.267, 1.267854785222098 + 0.6 * 2.718) node[right]{$v(t)=\left[ -r\omega \sin{\omega t}, r\omega \cos{\omega t}\right]$};
\end{tikzpicture}

\section{Acceleration}

We can get the acceleration by differentiating the components of the velocity vector.

$$a(t) = \left[-r \omega^2 \cos{\omega t}, -r \omega^2 \sin{\omega t} \right]$$

Notice that the acceleration vector points toward the center of the
circle it is traveling on.  That is, when an object is traveling on a
circle at a constant speed, its only acceleration is toward the center
of the circle.

\begin{tikzpicture}[declare function={angle=25;},bullet/.style={inner
    sep=1pt,fill,draw,circle,solid}, scale=1.7]
    % Axis
    \draw[thick,-stealth,black] (-3.2,0)--(3.2,0) node[right] {$x$}; % x axis
    \draw[thick,-stealth,black] (0,-3.2)--(0,3.2) node[left] {$y$}; % y axis
    % Rest
    \draw [dashed, sdkblue] (0,0) circle (3);
    \filldraw[black] (angle:3.0) circle(4pt);
    \draw[->, thick] (2.71892336110995, 1.267854785222098) --
    (2.71892336110995 * 0.2 , 1.267854785222098 * 0.2) node[midway, right]{$a(t) = \left[-r \omega^2 \cos{\omega t}, -r \omega^2 \sin{\omega t} \right]$};
\end{tikzpicture}

The magnitude of the acceleration vector is $r \omega^2$.

\section{Centripetal force}

How hard is the ball pulling against your hand?  That is, if you let
go, the ball would fly in a straight line.  The force you are exerting
on the string is what causes it to accelerate toward the center of the
circle. We call this the \newterm{centripetal force}.

Recall that $F = m a$.  The magnitude of the acceleration is $r
\omega^2 = 3 \left(\frac{2 pi}{4}\right)^2 \approx 7.4$ m/s.  The mass
of the ball is 0.16 kg.  So the force pulling against your hand is
about 1.18 newtons.

The general rule is that when something is traveling in a circle at a
constant speed, the centripetal force needed to keep it traveling in a
circle is:

$$F = m r \omega^2$$

If you know the radius $r$ and the speed $v$ of the object, here is the rule:

$$F = \frac{m v^2}{r}$$

\begin{Exercise}[title={Circular Motion}, label=circular]
Just as your car rolls onto a circular track with a radius of 200 m,
you realize your 0.4 kg cup of coffee is on the slippery dashboard of your
car.  While driving 120 km/hour, you hold the cup to keep it from sliding.

What is the maximum amount of force you would need to use (The friction of
the dashboard helps you, but the max is when the friction is zero.)

\end{Exercise}
\begin{Answer}[ref=circular]
  $$\frac{120 \text{ km}}{1 hour} = \frac{1000 \text{ m}}{1 \text{ km}}\frac{120 \text{ km}}{1 hour} \frac{1 \text{ hour}}{3600 \text{ seconds}}= 33.3 \text{ m/s}$$

  $$F = \frac{m v^2}{r} = \frac {0.4 (33.3)^2}{200} = 2.2 \text{ newtons}$$
\end{Answer}
