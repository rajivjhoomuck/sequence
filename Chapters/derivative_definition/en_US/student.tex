\chapter{Derivatives}

In calculus, the derivative of a function represents the rate at which
the function is changing at a particular point. It is a fundamental
concept that has vast applications in various fields, including
physics.\index{derivative}

\section{Definition}

The derivative of a function $f(x)$ at a point $x$ is defined as the limit:

\begin{equation}
f'(x) = \lim_{{h \to 0}} \frac{f(x+h) - f(x)}{h}
\end{equation}

provided this limit exists. In words, the derivative of $f$ at $x$ is
the limit of the rate of change of $f$ at $x$ as the change in $x$
approaches zero.

\section{Applications in Physics}

In physics, derivatives play a vital role in describing how quantities
change with respect to one another.

\subsection{Velocity and Acceleration}

In kinematics, the derivative of the position function with respect to
time gives the velocity function, and further taking the derivative of
the velocity function gives the acceleration function. For example, if
$s(t)$ represents the position of an object at time $t$, then the
velocity $v(t)$ and acceleration $a(t)$ are given by:

\begin{equation}
v(t) = \frac{ds}{dt} \quad \text{and} \quad a(t) = \frac{dv}{dt} = \frac{d^2s}{dt^2}
\end{equation}

\subsection{Force and Momentum}

In mechanics, the derivative of the momentum of an object with respect
to time gives the net force acting on the object, as stated by
Newton's second law of motion:

\begin{equation}
F = \frac{dp}{dt}
\end{equation}

where $F$ is the force, $p$ is the momentum, and $t$ is the time.

