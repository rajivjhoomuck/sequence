\chapter{Rules for Finding Derivatives}


Derivatives play a key role in calculus, providing us with a means of
calculating rates of change and the slopes of curves. Here, we present
some common rules used to calculate derivatives.

\section{Constant Rule}

The derivative of a constant is zero. If $c$ is a constant and $x$ is
a variable, then:\index{constant rule}

\begin{equation}
\frac{d}{dx}c = 0
\end{equation}

\section{Power Rule}

For any real number $n$, the derivative of $x^n$ is:\index{power rule}

\begin{equation}
\frac{d}{dx}x^n = nx^{n-1}
\end{equation}

\section{Product Rule}

The derivative of the product of two functions is:\index{product rule}

\begin{equation}
\frac{d}{dx}(fg) = f'g + fg'
\end{equation}

where $f'$ and $g'$ denote the derivatives of $f$ and $g$,
respectively.

\section{Quotient Rule}

The derivative of the quotient of two functions is:\index{quotient rule}

\begin{equation}
\frac{d}{dx}\left(\frac{f}{g}\right) = \frac{f'g - fg'}{g^2}
\end{equation}

\section{Chain Rule}

The derivative of a composition of functions is:\index{chain rule}

\begin{equation}
\frac{d}{dx}(f(g(x))) = f'(g(x)) \cdot g'(x)
\end{equation}

\section{Conclusion}

These rules form the basis for calculating derivatives in
calculus. Many more complex rules and techniques are built upon these
fundamental rules.
