\chapter{Differential Equations}

Differential equations are equations involving an unknown function and
its derivatives. They play a crucial role in mathematics, physics,
engineering, economics, and other disciplines due to their ability to
describe change over time or in response to changing conditions.\index{differential equations}

\section{Ordinary Differential Equations}

An ordinary differential equation (ODE) involves a function of a
single independent variable and its derivatives. The order of an ODE
is determined by the order of the highest derivative present in the
equation. An example of a first-order ODE is:\index{ordinary
  differential equation} \index{ODEs}

\begin{equation}
\frac{dy}{dx} + y = x
\end{equation}

Here, $y$ is the function of the independent variable $x$, and $\frac{dy}{dx}$ represents its first derivative.

\section{Partial Differential Equations}

Partial differential equations (PDEs), on the other hand, involve a
function of multiple independent variables and their partial
derivatives. An example of a PDE is the heat equation, a second-order
PDE:\index{partial differential equations} \index{PDEs}

\begin{equation}
\frac{\partial u}{\partial t} = \alpha \frac{\partial^2 u}{\partial x^2}
\end{equation}

In this equation, $u = u(x, t)$ is a function of the two independent
variables $x$ and $t$, $\frac{\partial u}{\partial t}$ is the first
partial derivative of $u$ with respect to $t$, and $\frac{\partial^2
  u}{\partial x^2}$ is the second partial derivative of $u$ with
respect to $x$.


