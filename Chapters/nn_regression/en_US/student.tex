\chapter{Neural Nets for Regression}

Neural Networks are powerful computational models that are used for a
variety of tasks in machine learning, including regression. Unlike
linear regression, which is a linear approach to modelling the
relationship between a dependent variable and one or more independent
variables, a Neural Network can model complex non-linear
relationships.\index{neural net regression}

\section{Neural Networks}

A neural network is composed of nodes (neurons) grouped into
layers. The input layer takes in the features of the dataset, hidden
layers perform computations on these inputs, and the output layer
produces the final prediction. Each node in a layer is connected to
each node in the next layer through "edges". These edges carry
weights, which are the parameters of the model that are learned during
training.

A key part of each node is an activation function. It transforms the
weighted sum of the node's inputs into an output value that is passed
onto the next layer. Common activation functions include the ReLU
(Rectified Linear Unit), sigmoid and hyperbolic tangent functions. For
regression tasks, usually a linear activation function is used in the
output layer, as the output can be any real number.

\section{Neural Networks for Regression}

To perform regression using a neural network, you would train the
network to map the input features to a continuous target variable.

Here is the basic process:

\begin{itemize}
\item \textbf{Feedforward:} Compute the output of the network given
  the input features. This involves calculating the weighted sum of
  inputs for each node and applying the activation function.

\item \textbf{Loss Calculation:} Calculate the loss (difference
  between the network's prediction and the actual value). For
  regression tasks, common loss functions include Mean Squared Error
  (MSE) or Mean Absolute Error (MAE).

\item \textbf{Backpropagation:} Update the network's weights to
  minimize the loss. This is done by computing the gradient of the
  loss function with respect to each weight in the network, and then
  adjusting the weights in the direction that decreases the loss.

\item \textbf{Iteration:} Repeat the feedforward, loss calculation,
  and backpropagation steps for a number of epochs (complete passes
  through the dataset) or until the loss converges to a minimum.

\end{itemize}

This way, the neural network learns to approximate the function that
best maps input features to the target variable, thus performing
regression. The advantage of using neural networks over linear
regression is that they can capture complex non-linear relationships
between variables.
