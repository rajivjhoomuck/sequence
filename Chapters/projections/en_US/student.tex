\chapter{Projections}

The projection of a vector $\mathbf{a}$ onto another vector
$\mathbf{b}$ is a vector that lies along $\mathbf{b}$ and represents
the component of $\mathbf{a}$ in the direction of $\mathbf{b}$. This
projection can be computed using the dot product of the two vectors.\index{projection}

Given vectors $\mathbf{a}$ and $\mathbf{b}$, the projection of
$\mathbf{a}$ onto $\mathbf{b}$, denoted as
$\mathbf{proj}_\mathbf{b}(\mathbf{a})$, can be calculated as follows:

$$\mathbf{proj}_\mathbf{b}(\mathbf{a}) = \left(\frac{{\mathbf{a} \cdot \mathbf{b}}}{{\|\mathbf{b}\|^2}}\right) \mathbf{b}$$

where $\cdot$ denotes the dot product and $|\mathbf{b}|$ represents
the magnitude (or length) of vector $\mathbf{b}$.

The numerator $\mathbf{a} \cdot \mathbf{b}$ measures the extent to
which $\mathbf{a}$ and $\mathbf{b}$ are aligned with each
other. Dividing this by $|\mathbf{b}|^2$ scales the projection to
ensure it represents the correct length along $\mathbf{b}$.

Finally, multiplying the scaled value with $\mathbf{b}$ gives us the
projection vector itself.

In summary, the dot product is used to determine the alignment between
two vectors, and by appropriately scaling one vector and multiplying
it with the other vector, we can obtain the projection of one vector
onto the other.
