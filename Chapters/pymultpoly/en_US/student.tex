\chapter{Multiplying Polynomials in Python}

At this point, you have created a nice toolbox of functions for
dealing with lists of coefficients as polynomials. Create a file called \filename{poly.py} and copy the folowing functions into it:
\begin{itemize}
\item \pyfunction{evaluate\_polynomial}
\item \pyfunction{polynomial\_to\_string}
\item \pyfunction{add\_polynomials}
\item \pyfunction{scalar\_polynomial\_multiply}
\item \pyfunction{subtract\_polynomial}
\end{itemize}

Now create another file in the same directory called \filename{test.py}. Type this into that file:
\begin{Verbatim}
import poly

polynomial_a = [9.0, -4.0, 3.0, -5.0]
print('Polynomial A =', poly.polynomial_to_string(polynomial_a))

polynomial_b = [-9.0, 0.0, 4.0, 2.0, 1.0]
print('Polynomial B =', poly.polynomial_to_string(polynomial_b))

# Evaluation
value_of_b = poly.evaluate_polynomial(polynomial_b, 3)
print('Polynomial B at 3 =', value_of_b)

# Adding
a_plus_b = poly.add_polynomials(polynomial_a, polynomial_b)
print('A + B =', poly.polynomial_to_string(a_plus_b))

# Scalar multiplication
b_scalar = poly.scalar_polynomial_multiply(-3.2, polynomial_b)
print('-3.2 * Polynomial B =', poly.polynomial_to_string(b_scalar))

# Subtraction
a_minus_b = poly.subtract_polynomial(polynomial_a, polynomial_b)
print('A - B =', poly.polynomial_to_string(a_minus_b))
\end{Verbatim}

When you run it, you should get the following:
\begin{Verbatim}
Polynomial A = -5.0x^3 + 3.0x^2 + -4.0x + 9.0
Polynomial B = 1.0x^4 + 2.0x^3 + 4.0x^2 + -9.0
Polynomial B at 3 = 162.0
A + B = 1.0x^4 + -3.0x^3 + 7.0x^2 + -4.0x
-3.2 * Polynomial B = -3.2x^4 + -6.4x^3 + -12.8x^2 + 28.8
A - B = -1.0x^4 + -7.0x^3 + -1.0x^2 + -4.0x + 18.0
\end{Verbatim}

Now you are ready to implement multiplication of polynomials. The function will look like this:
\begin{Verbatim}
def multiply_polynomials(a, b):
  ...Your code here...
\end{Verbatim}
It will return a list of coefficients.

In an exercise in the last chapter, you were asked `` Let's say I have
two polynomials, $p_1$ and $p_2$.  $p_1$ has degree 23.  $p_2$ has
degree 12.  What is the degree of their product?'' The answer was $23 +
12 = 35$.

In our implementation, a polynomial of degree 23 is held in a list of length 24.

In Python we wil be trying to multiply a polynomial $a$ and a
polynomial $b$ represented as lists. What is the degree of that product?
\begin{Verbatim}
      result_degree = (len(a) - 1) + (len(b) - 1)
\end{Verbatim}

Now, we need to create an array of zeros that is one longer than that. Here is a cute Python trick: if you have a list, you can replicate it using the * operator. 
\begin{Verbatim}
a = [5,7]
b = a * 4
print(b)
# [5, 7, 5, 7, 5, 7, 5, 7]
\end{Verbatim}

Here's how you will get a list of zeros:
\begin{Verbatim}
      result = [0.0] * (result_degree + 1)
\end{Verbatim}

We will step through $a$ getting the index and value of each entry. You can do this in one line using \pyfunction{enumerate}:
\begin{Verbatim}
      for a_degree, a_coefficient in enumerate(a):
\end{Verbatim}
For each of those, we will step through the entire $b$ polynomial. As
you multiply together each term, you will add it to appropriate
coefficient of the result.

Here is the whole function:
\begin{Verbatim}
def multiply_polynomials(a, b): # What is the degree of the resulting
polynomial?  result_degree = (len(a) - 1) + (len(b) - 1)

    # Make a list of zeros to hold the coefficents result = [0.0] *
    (result_degree + 1)

    # Iterate over the indices and values of a for a_degree,
    a_coefficient in enumerate(a):

        # Iterate over the indices and values of b for b_degree,
        b_coefficient in enumerate(b):

            # Calculate the resulting monomial coefficient =
            a_coefficient * b_coefficient degree = a_degree + b_degree
            
            # Add it to the right bucket
            result[degree] = result[degree] + coefficient
            
    return result
\end{Verbatim}

Take a long look at that function.  When you understand it, type it into \filename{poly.py}.

In \filename{test.py}, try out the new function:
\begin{Verbatim}
# Multiplication
a_times_b = poly.multiply_polynomials(polynomial_a, polynomial_b)
print('A x B =', poly.polynomial_to_string(a_times_b))
\end{Verbatim}

This is an example of a \emph{nested loop}. The outer loop steps
through the polynomial $a$. For each step it takes, the inner loop
steps through the entire polynomial $b$.

\section{Something surprising about lists}

You can imagine that you might want to create two very similar polynomials. Let's say polynomial $c$ is $x^2 + 2x + 1$ and polynomial $d$ is $x^2 -2x + 1$.  You might think you are very clever to just alter that degree 1 coefficient like this:
\begin{Verbatim}
c = [1.0, 2.0, 1.0]
d = c
d[1] = -2.0
\end{Verbatim}

If you printed out $c$, you would get $[1.0, -2.0, 1.0]$.  Why? You
assigned two variables ($c$ and $d$) to the \emph{the same list}.  So
when you use one reference ($d$) to change the list, you see the
change if you look at the list from either reference. \emph{FIXME:
  Diagram of two references to the same list here.}

To create two separate lists, you would need to explicitly make a copy:
\begin{Verbatim}
c = [1.0, 2.0, 1.0]
d = c.copy()
d[1] = -2.0
\end{Verbatim}

