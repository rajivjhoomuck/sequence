\chapter{u-Substitution}


U-Substitution, also known as the method of substitution, is a technique used to simplify the process of finding antiderivatives and integrals of complicated functions. The method is similar to the chain rule for differentiation in reverse.

Suppose we have an integral of the form:

\begin{equation}
\int f(g(x)) \cdot g'(x) \, dx
\end{equation}

The u-substitution method suggests letting a new variable $u$ equal to the inside function $g(x)$, i.e., 

\begin{equation}
u = g(x)
\end{equation}

Then, the differential of $u$, $du$, is given by:

\begin{equation}
du = g'(x) \, dx
\end{equation}

Substituting $u$ and $du$ back into the integral gives us a simpler integral:

\begin{equation}
\int f(u) \, du
\end{equation}

This new integral can often be simpler to evaluate. Once the antiderivative of $f(u)$ is found, we can substitute $u=g(x)$ back into the antiderivative to get the antiderivative of the original function in terms of $x$.

The method of u-substitution is a powerful tool for evaluating integrals, especially when combined with other techniques like integration by parts, partial fractions, and trigonometric substitutions.
